%*******************************************************
% Abstract
%*******************************************************
\pdfbookmark[1]{Abstract}{Abstract}
\chapter*{Abstract}

X-ray computed tomography (XCT) is beginning to find a range of new industrial applications. For many years this technique has been applied to non-destructive testing (NDT), however it is now being used by industry for the purpose of metrology. XCT has certain advantages over the traditional approach to metrology that of the coordinate measuring machine (CMM). Foremost among these is the ability to measure both external and internal aspects of an objects geometry without the need to destroy the object. This technique does however suffer from the lack of a clear understanding of the processes metrological uncertainties - if XCT is to be adopted more widely then it is necessary to be able to quantify the underlying uncertainties in this measuring procedure. This Thesis will approach the problem of quantifying these uncertainties via the simulation of an XCT system. The focus will be on the effect of magnification on measurement uncertainty in the presence of realistically modeled source and detector elements.

\textbf{\textit{A previous version of parts of this text was submitted as part of FEEG6018}}
